\chapter{Comparison of Other Guides}
\label{chap:comparison}

Once the collection of knowledge done, our next step is to compare other guides that allow assessments on \gls{ict} systems. This step will allow us to explore multiple approaches with their own characteristics and usages, which will bring useful information for our proposal that will be defined later in \fullrefnametype{chap:proposal}.

We will start by defining the methodology that will be applied in order to find, analyse and compare the various guides. Then, we will apply this methodology which will lead to a comparison of those methods based on their analysis. Finally, a summary will be made on the whole Chapter content.
\minitoc

\newpage

% -----------------------------------------------------------------------------
\section{Methodology}
\label{sec:comparison_methodology}

A methodology has been defined to help us analyse and compare the most appropriate guides close to our scope. We designed it in a simple way while ensuring that an accurate and coherent comparison can be carried out.

\subsection{Selection}
\label{subsec:comparison_selection}

We need to select guides that are well recognized and relevant in the security and privacy fields. They should ideally be specialized for web services and should also integrate \gls{ai}-related concerns. We found citations of such guides during our knowledge collection made in \fullrefnametype{subsec:state_review_results}.

Their format is not fixed; it could be a checklist, an evaluation, a \gls{framework}, an online tool, et cetera. It could be issued by any organization, but its expertise must be established.

We will select the most adapted guides in our opinion, based on facts, legitimacy and general recognition.

\subsection{Characteristics}
\label{subsec:comparison_characteristics}

Some characteristics will be extracted from the selected guides in order to expose and compare their approaches, features and objectives. Those characteristics can be exclusive to a guide or common for some of them. 

The definition of the characteristics have been defined by taking our thesis scope and objectives into account.
\begin{itemize}
    \item \textbf{Accessible}: is the guide easy and light to use?
    \item \textbf{\gls{ai}-oriented}: does the guide include specific issues coming from the \gls{ai} field? If so, are those issues sufficient?
    \item \textbf{Format}: how is the guide used? For example, by going through checklists or sets of directives.
    \item \textbf{Scoring}: does the guide provide a final score to represent the security and/or privacy levels?
    \item \textbf{Volume}: is the guide workload heavy for a small or medium-sized organization?
    \item \textbf{Web-orientated}: does the guide include specific issues coming from the web environment? If so, are those issues sufficient?
\end{itemize}

If some useful and accurate characteristics are found during the analyses, we will include them in the comparison process as well.

\subsection{Assessment}
\label{subsec:comparison_assessment}

The assessment of the guides will consist of listing them and comparing their characteristics. The goal of this comparison is to get an understandable and clear overview of the guides we found. Their analysis will allow us to present a factual, unbiased and complete comparison based on their said characteristics. This comparison will be useful to define our own defined guide.

\section{Guides Analysis}
\label{sec:comparison_analyse}

The following Subsections contain an analysis of their respective guide.

\subsection{\gls*{nist} Cybersecurity \Gls*{framework}}
\label{subsec:comparison_analyse_nist}

The \citeproperref{\gls{nist} Cybersecurity \Gls*{framework}}{https://www.nist.gov/cyberframework}{2022}{11}{18} has been developed with the objective of \say{Helping organizations to better understand and improve their management of cybersecurity risk}. Its format is a \gls{pdf} file which can be found online on the \gls{nist} official website. 

This \gls{framework} is organized into five categories: \textit{identify}, \textit{protect}, \textit{detect}, \textit{respond}, and \textit{recover}. Those categories themselves have additional categories, subcategories and references to go into specifics. It includes standards, guidelines, and practices that are useful to guide cybersecurity activities. It is focused on multiple areas which are the \gls{ict} field, industrial control systems, cyberphysical systems and the \gls{iot} topic. This tool can be used in various ways, but is oriented towards risk management practices analysis.

This tool integrates the possibility to create \citeproper{profiles}, which are roadmaps that include some requirements for various specific sector goals. It enables organizations to describe their current and desired targeted states, which can also be shared across entities.

There is no evaluation or scoring capabilities with this tool, its users can only consult the descriptions of items and use them as guidance to assess their systems.

Alongside of the \gls{framework} itself, \gls{nist} has published a quick start guide to help organizations to use their tool by explaining the different categories with additional information. They also provide an online learning platform to answer some frequently asked questions and to show various examples.

This \gls{framework} does not guarantee a complete and exhaustive guidance, but it does try to be as complete as possible regarding cybersecurity management in the previously mentioned areas. No particular assessment process is given to use this tool, which can be confusing for its users.

Its usage is quite complex, and it has a lot of content. A lack of technical advices and implementation details can be noticed: it is mostly designed for managers than for developers. However, the \gls{framework} content is quite expressive and is well referenced.

No item is specifically oriented towards \gls{ai} processes nor web services. Furthermore, this tool is specialized into cybersecurity and does not include dedicated concerns about privacy.

\subsection{\gls*{ncsc} \acrlong*{caf}}

The \citeproperref{\gls{ncsc} \gls{caf}}{http://bit.ly/3AxqEs7}{2022}{11}{18} tool aims to guide organizations to assess which entity is responsible in terms of security risks. It defines four objectives: \textit{managing security risks}, \textit{protecting against cyberattacks}, \textit{detecting cybersecurity events}, and \textit{minimizing the impact of cybersecurity incidents}. Each objective includes principles of specific security aspects which list and explain various rules to be respected.

This guide can be accessed online by browsing the principles, which are also categorized and indicated by a letter depending on the objective they cover. Each rule that a principle includes is defined with a title and its explanation. No scored evaluation can be conducted, but the rules are described in a way that allows to verify the assessed system compliance with them.

No specificities are given for \gls{ai} processes concerns, neither for user privacy ones. Furthermore, this tool is not oriented towards web services. However, each principle can be overviewed in tables that efficiently summarize how to be compliant to each given rule.

A notation based on the compliance level for each rule is given, which can be either achieved, not achieved, and sometimes partially achieved. Levels are determined by the compliance observed within the organization by evaluating the description of each level contained in rules.

Additional online resources are given to explain and help on how to use the guide. They are available on the same website as the tool. A complete guidance can also be consulted.

The \gls{caf} is somewhat heavy to approximate because of its verbosity. Moreover, the rules descriptions can be complex to assess, with a lot of details. However, we noticed that this tool is quite complete in its coverage. It also provides additional and multiple resources for each principle, such as other relevant guides.

\subsection{\citeproper{ENISA} Resources}

\citeproperref{ENISA}{https://enisa.europa.eu}{2022}{11}{18} is the European Union agency for cybersecurity whose activity is to spread awareness about cybersecurity across Europe. They do not provide any complete tool for \gls{ict} systems security and privacy assessment in their entirety, but they do publish multiple projects for organizations on specific scopes. We selected two of them that best suit our scope: The \citeproperref{Risk Level Tool}{http://bit.ly/3Xqkkwi}{2022}{11}{19} and the \citeproperref{SecureSME project}{http://bit.ly/3EApptm}{2022}{11}{19}. The latter includes various guides that aim to raise the cybersecurity levels in organizations.

The \citeproper{Risk Level Tool} evaluates risk levels of a personal data processing operation by doing a security risk assessment. The evaluation is composed of five steps: \textit{definition and context}, \textit{impact evaluation}, \textit{threat analysis}, \textit{risk evaluation} and \textit{security measures}. It is doable online and provides a list of security measures with their related risk level based on the form previously filled. Measures are given for each category of security issues listed in tables.

This tool is highly specialized on privacy concerns and has been designed for the organizations decision makers, without providing any technical details. It is quite complete and therefore long to conduct, however its online availability and simple navigation makes the tool accessible given the complex field of risk management.

Once filled, the result can then be shared, printed or saved. No overall score of the risk level is given.

The \citeproper{SecureSME project} includes several specialized guides. They are all different, but they all share the common goal to help organizations to secure their \gls{ict} systems. They are focused on three categories, which are employees' protection, process enhancement and reinforcements of technical measures. The guides can be written in English, French or German, because each one of them is produced by local European agencies.

A summarized version of the guides named \citeproper{Cyber Tips} can be found online: it lists questions for each category that must be answered to reach higher security levels. This resource is useful to get an overview of the \citeproper{ENISA} guides in a lighter way. However, no guidance or details are given to go further.

The \citeproper{Cybersecurity guide for SMEs - 12 steps to securing your business} leaflet provides twelve high level steps to help organizations to secure their systems. It is based on the \citeproper{Cybersecurity for SMEs - Challenges and Recommendations} report, which contains more details and specifics. Each step has a single or multiple advices that can be applied in order to enhance security, with a small description of them.

The \citeproper{Cybersecurity Guide for SME} is an equivalent of the leaflet above: it also lists various categories that includes multiple items to be compliant with to raise security levels. Each item are explained by small descriptions.

Other resources exist for particular topics such as COVID-19 concerns or supply chain attacks risks: yet, none of them are useful for a generic guidance. Overall, no resource is web-oriented or includes \gls{ai} processes. Apart for the risk assessment, no evaluation can be done on systems.
 
The resources we analysed are quite accessible and easy to understand, but they are shallow and have a lack of specifics. They are useful for simple guidances, but the \citeproper{ENISA} project is harder to browse and to understand as a whole than the other guides that have a unique method and guide.

\subsection{\gls*{cisa} Cyber Essentials \Glspl{toolkit}}

The \citeproperref{\gls{cisa} Cyber Essentials \Glspl{toolkit}}{http://bit.ly/3GOf3bY}{2022}{11}{23} is a collection of actions designed to help organizations to bring cybersecurity in their processes. Each \gls{toolkit} is specialized in a specific area of organizations: \textit{leader}, \textit{users}, \textit{systems}, \textit{surroundings}, \textit{data} and \textit{crisis response}.

The \glspl{toolkit} are presented on two \gls{pdf} pages using a user-friendly and well-presented design. A generic task is defined to improve security levels, with a linked set of actions that must either be applied by leaders, \gls{ict} staff or service providers. The actions contain a brief description and give additional resources to take action.

One of the \glspl{toolkit} gives an action that directly concerns data privacy issues, but nothing generic or centred on user privacy is given in the other ones. Moreover, neither the web area nor \gls{ai} processes are mentioned or taken into consideration.

This tool is not intended to be used as a guide, just as an entry point to provide further resources that can be used to improve cybersecurity levels by adequate actions.

Few rules are given: the \glspl{toolkit} have been designed to be short and concise. However, they are easy to use and to understand thanks to their brevity. 

Some webinars on this tool are available on \citeproper{Youtube}, which can be useful when actions need to be taken in an organization. A small overview of each area of specialization including some advices is also provided on the \glspl{toolkit} website.

\section{Guides Comparison}
\label{sec:comparison_comparison}

As explained in \fullrefnametype{sec:comparison_methodology}, a comparison between the resources has to be realized. The final results are shown in \fullrefnametype{table:comparison_evaluation}. Each guide name has been shortened to the title of their respective organization for readability purpose. 

\begin{table}[ht]
    \begin{center}
        \begin{tabular}{l|cccc}
            \toprule[0.8mm]
            \textbf{Characteristic} & \gls{nist} & \gls{ncsc} & \citeproper{ENISA} & \gls{cisa} \\ 
            \midrule[0.8mm]
            Accessible  & \cmark & $\pmb{\sim}$ & \xmark & \cmark \\
            \gls{ai}    & \xmark & \xmark & \xmark & \xmark \\
            Format      & list of items & tables of rules & multiple & set of actions \\
            Privacy     & \xmark & \xmark & $\pmb{\sim}$ & \xmark \\
            Scoring     & \xmark & compliance levels & risk matrix & \xmark \\
            Volume      & high & high & medium & low \\
            Web         & \xmark & \xmark & \xmark & \xmark \\
            \bottomrule[0.8mm]
        \end{tabular}
		\tableannotations{-1.25cm}{\cmark: fills the criteria; \xmark: does not fill the criteria; $\pmb{\sim}$: almost fills the criteria.}
    \end{center}
    \caption{Comparison of the selected guides based on their characteristics}
    \label{table:comparison_evaluation}
\end{table}

We can see that for all the guides we analysed, none of them include specific content focused on \gls{ai} processes. Furthermore, technologies used by web services are not represented, despite their wide adoption.

The format of the guides varies a lot between them. Their volume also changes according to their diverse workload. The \citeproper{ENISA} resources and \gls{cisa} \Glspl{toolkit} are the lightest guides, with the latter being the most accessible one.

The resources have a lack of scoring capacities, which would be useful for developers and decision makers whom wish to get an overall score of their security and privacy levels. Furthermore, getting scores would allow to prioritize the most sensitives parts of an evaluated service.

Finally, the guides we found are focused on security issues only. We noticed some actions oriented towards data privacy in one of the \citeproper{ENISA} resources, but nothing has been specified in terms of user privacy. Hence, we added the \citeproper{privacy} characteristic into the comparison table. This lack could be explained by the fact that privacy in \gls{ict} systems is considered as a separate concern from security. But in our opinion, those two concerns are closely related, and additional serious issues can appear from weak privacy levels.

% -----------------------------------------------------------------------------
\section{Summary}
\label{sec:comparison_summary}

This Chapter has presented an analysis and a comparison of multiple guides close to our thesis scope. The organizations that created them represent the countries that have the stronger impact and capacities in the security domain and cybersecurity surveillance: the United States of America, the United Kingdom, and the European Union.

We discovered that the guides have various characteristics, but none of them have integrated specific risks and issues from the privacy field, the \gls{ai} processes field or the web environments in their content. This observation is concerning because of the recognition of each of the organizations that created those guides: indeed, a lot of \gls{ict} decision makers and developers follow their recommendation, which could lead to a lack of security and privacy levels for those specific three fields. There is therefore a lack of consideration towards lots of major risks, which we can contribute to fill with a new proposal.

The said new proposal should bring additional considerations to the previously identified lacks, alongside a coverage on the more classic \gls{ict} security issues. Moreover, its format, volume and accessibility should be optimized and designed to simplify its adoption and usage. By doing so, this proposal would meet all the characteristic we defined, which means that it would bring a strong added value for developers and \gls{ict} decision makers.