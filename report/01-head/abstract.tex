\chapter*{Abstract}
\addcontentsline{toc}{chapter}{Abstract}

The vast majority of Internet users around the world accesses web services, for both personal and professional goals, often involving sensitive information. Therefore, web services must be secure to prevent attacks resulting from vulnerabilities being exploited. Against this background, recent advances in data science and Artificial Intelligence (AI) have led to an increased adoption of personalized, adaptive and customizable models in many areas, including in web services. However, these models need vast amounts of user data to achieve a satisfactory performance,  raising important privacy concerns. Therefore, it is critical for organizations to ensure that the web services they provide are both developed securely and designed respectfully of the privacy of their end users.  

This thesis addresses such need by providing an accessible, simple, and complete guide to evaluate web services for organizations. To meet this goal, we conducted a state-of-the-art review on multiple \acrlong*{ict} fields related to both the web environment and big data, focusing on security and privacy concerns. In addition, an evaluation of reputable existing guides (by NIST, NCSC, ENISA and CISA) was conducted to identify the gaps regarding web services, which informed the scope  for the development of this new guide. Several guidelines were then developed to allow organizations to assess whether their web services are compliant with recommendations, mitigations or other useful information, incorporating the  knowledge collected. Finally, an online web application has been developed specifically for a dynamic assessment of web services by organizations, against the entire knowledge collection we made in an easy, complete, efficient and accessible manner. The application, hosted at \href{https://ohmygasp.com}{ohmygasp.com}, allows organizations to receive heuristic scores  on the security and privacy levels of their assessed web services, on multiple categories. Those scores and the requirement levels specified for the knowledge collection are provided in order to give the organization a prioritized list on the most sensitive concerns, in order to quickly activate levers to strengthen their web services. The accuracy of our proposal and guide has been reviewed by assessing an existing web service, which suggests that our approach is not only valid  but also useful.

In its present form, the use of this guide already can benefit any organization developing web services. However, in addition, its generic design allows it to be extended for evaluations of systems from other contexts using customized guide contents, and therefore having the potential to become a centralized or decentralized hub that could store, compare and help design  guide contents for various subjects, scopes or technologies. This growth would allow us to improve the security and privacy levels of many more systems.

\vskip0.5cm
\textbf{Keywords:} 
\Keywords