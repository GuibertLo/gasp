\chapter{Context}
\label{chap:context}

This chapter will describe the context around the subject of the thesis. We will explain the actual state that contains a problem to be solved, and the author's contribution that will contribute to fix the previously described problem. We will also introduce the different actors that will intervene during the thesis.

% -----------------------------------------------------------------------------
\section{Actual State}
\label{sec:context_actual}

In a world where \gls{it} security is increasingly valuable and necessary, the need for new ways to secure and trust information systems is growing. This need is particularly expressed in the \gls{ai} field, where big amounts of data are periodically collected in order to improve services performances or to monetize them. Furthermore, such data is often personal and highly related to their user, which raise ethical and privacy-related questions.

Nowadays, end users of public or private online services are more and more aware of personal data related risks and a change in consumption patterns is being noticed. Therefore, new approaches for the whole \gls{it} field must be developed in order to provide secured and privacy-first online services that can nevertheless enable a personalized experience.

% -----------------------------------------------------------------------------
\section{Contribution}
\label{sec:context_contribution}

By enabling secured and privacy-oriented personalized experience on online services, companies would be able to provide ethical, modern and respectful offers to their customers. All stakeholders would benefit from such implementations, as long as the performance, time or processing capabilities do not restraint them in their activities.

This thesis aims to provide which technologies, best practices or safeguards can be integrated to information systems in order to ensure secured environments to the end users, particularly regarding AI projects. These components must then be implemented in a functional information system, which includes \gls{ai} processes, while providing a conclusion on the changes of such integration compared to the initial information system.

We want to offer a suitable solution to those wishing to increase the security and confidentiality of their online services. A focus will also be made on the \gls{ai} field.

% -----------------------------------------------------------------------------
\section{Actors}
\label{sec:context_actors}

The thesis will be conducted by \Author.

There are two advisors for this thesis: \AdvisorOne \ and \AdvisorTwo. 

The thesis subject has been proposed by \Author \ as a personal project and is related to the field of research of the two advisors. A secured online service is being developed by \AdvisorOne \ and includes aspects similar to this thesis subject.

An expert will be assigned to the thesis in order to evaluate it when completed and returned. This assignment will be made later during the semester.